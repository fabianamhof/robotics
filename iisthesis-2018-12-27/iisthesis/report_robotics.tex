\documentclass[report]{iisthesis}
%           or master (bachelor or master is required)

% These two packages are highly recommended:
\usepackage[T1]{fontenc} % make non-ASCII characters cut&pastable in PDF
\usepackage{lmodern}     % easiest way to get outline fonts with T1 encoding

% \usepackage[ngerman]{babel}     % if the thesis is written in German

\title{IIS Thesis Instructions}
\author{Fabian Amhof \\ Matteo Quaratino}
\supervisor{Dr. Matteo Saveriano}
%\supervisor{Firstname1 Lastname1\\ Firstname2 Lastname2}


\begin{document}
\maketitle
\tableofcontents

\chapter{Introduction}
\chapter{Improvements}
\label{improvements}

\section{Control Robot by ROS-node on external machine}
\label{separate_node}
We propose to control the robot with a ROS-node on a external machine. 


\section{Estimate Table Velocity}
\label{estimate_velocity}
In the example we were given the robot arm does not adjust its movement in relation to the speed at which the table 
turns. The timing and movement speed is hardcoded, and so is the speed of the table. \\
We want to work on this by letting the arm adjust its movement dynamically to the table speed.

\section{Improve Grabbing Capabillities}
\label{improve_grabbing}
The grabbing motion of the current implementation is reliant on the table turn speed being a specific and constant value, since it's hardcoded. If it varies even by the slightest bit
the arm could potenionally run into problems and for example push the cube off the table, ideally this should not happen at all. \\
We want it to react accordingly by building on \ref{estimate_velocity}, assuming that a faster cube is harder to grasp, it might be necessary to define another grabbing motion 
or at least change it for different situations, like a uneven speed of the table. \\
If all goes well, a option to reposition a misaligned cube could also be implemented.

\section{Implement RBT}
\label{implement_rbt}

\section{Use Camera to detect Cube [OPTIONAL]}
\label{use_camera}

\end{document}
